% This file contains all the necessary setup and commands to create
% the preliminary pages according to the buthesis.sty option.


\title{
    Tools for interfacing, extracting, and analysing neural signals using wide-field fluorescence imaging and optogenetics in awake behaving mice}

\author{Mark E. Bucklin}

\degree=2

\prevdegrees{B.S., Columbia University, 2009\\ M.S., Boston University, 2016}

\department{Department of Biomedical Engineering}

\defenseyear{2019} \degreeyear{2019}

\reader{First}{Xue Han, Ph.D.
}{% the title in with the name.% because it will overflow to a new page. You may try to put part of% Warning: If you have more than five readers you are out of luck,% or you will be asked to reprint and get new signatures.%   Professor, Department of Electrical and Computer Engineering"% or similar, but it MUST NOT be:%   "Professor of Electrical and Computer Engineering",% then name, and title. IMPORTANT: The title should be:% For each reader, specify appropriate label {First, Second, Third},% January of year X+1% will be in the fall of year X, and your graduation will be in% will be the same, except for January graduation, when your defense% the year the dissertation is written up and defended. Often, these% Degree year is the year the diploma is expected, and defense year is%   3 = Master of Science thesis and Doctor of Philisophy dissertation.%   2 = Doctor of Philisophy dissertation.%   1 = Master of Science thesis,% Type of document prepared for this degree:
    Associate Professor of Biomedical Engineering} \reader{Second}{David Boas, Ph.D.
}{
    Professor of Biomedical Engineering} \reader{Third}{Ian Davison, Ph.D.
}{
    Assistant Professor of Biology} \reader{Fourth}{Jerome C.
    Mertz, Ph.D.
}{
    Professor of Department of Biomedical Engineering} \reader{Fifth}{Kamal Sen, Ph.D.
}{
    Associate Professor of Biomedical Engineering}

\numadvisors=1 \majorprof{Xue Han, Ph.D.
}{{Associate Professor of Biomedical Engineering}}% (advisors) can be defined. % specified again for the abstract page. Up to 4 Major Professors% The Major Professor is the same as the first reader, but must be
%\majorprofb{First M. Last, PhD}{{Professor of Computer Science}}
%\majorprofc{First M. Last, PhD}{{Professor of Astronomy}}
%\majorprofd{First M. Last, PhD}{{Professor of Biomedical Engineering}}


%%%%%%%%%%%%%%%%%%%%%%%%%%%%%%%%%%%%%%%%%%%%%%%%%%%%%%%%%%%%%%%%

%                       PRELIMINARY PAGES
% According to the BU guide the preliminary pages consist of:
% title, copyright (optional), approval,  acknowledgments (opt.),
% abstract, preface (opt.), Table of contents, List of tables (if
% any), List of illustrations (if any). The \tableofcontents,
% \listoffigures, and \listoftables commands can be used in the
% appropriate places. For other things like preface, do it manually
% with something like \newpage\section*{Preface}.

% This is an additional page to print a boxed-in title, author name and
% degree statement so that they are visible through the opening in BU
% covers used for reports. This makes a nicely bound copy. Uncomment only
% if you are printing a hardcopy for such covers. Leave commented out
% when producing PDF for library submission.
%\buecethesistitleboxpage

% Make the titlepage based on the above information.  If you need
% something special and can't use the standard form, you can specify
% the exact text of the titlepage yourself.  Put it in a titlepage
% environment and leave blank lines where you want vertical space.
% The spaces will be adjusted to fill the entire page.
\maketitle
\cleardoublepage

% The copyright page is blank except for the notice at the bottom. You
% must provide your name in capitals.
\copyrightpage
\cleardoublepage

% Now include the approval page based on the readers information
\approvalpage
\cleardoublepage

% Here goes your favorite quote. This page is optional.
\newpage
%\thispagestyle{empty}
\phantom{.}
\vspace{4in}

\begin{singlespace}
	\begin{quote}
		\textit{
			We've got three flies, five flies;
		}\\
	\end{quote}
\end{singlespace}

% \vspace{0.7in}
%
% \noindent
% [The descent to Avernus is easy; the gate of Pluto stands open night
% and day; but to retrace one's steps and return to the upper air, that
% is the toil, that the difficulty.]

\cleardoublepage

\newpage
\begin{center}
\centerline{for pickle}
\end{center}

\cleardoublepage

\newpage
\hypertarget{preface}{%
\subsubsection*{Preface}\label{preface}}
\addcontentsline{toc}{subsubsection}{preface}

I have structured this document to roughly coincide with a chronological account of 6 years spent in a neuro-oriented biomedical engineering lab.
My role in the lab was centered around exploratory device design and development, mostly targeting application in neuroscience research, with intended users being neuroscientist colleagues.
One of the lab's most remarkable assets is the breadth and diversity of its constituents in terms of their skills and experience, both within and between the engineering/development and the science/medical sides of the lab.
All efforts stood to benefit from the close proximity to skilled colleagues, most notably for the complementary guide and provide roles that assisted the development process of new devices and the experiments they were intended for.

My initial experience in optoelectronic device development was as an undergrad at Columbia University where I was advised by Elizabeth Hillman, and developed a device that combined thermography and near-infrared spectroscopy in a portable and inexpensive device intended to provide early detection of adverse neoplastic changes through at-home daily monitoring, particularly targeting use by patients with high-risk for breast cancer.
I then went to the Das Lab where I developed macroscopic imaging systems used for intrinsic imaging in the visual cortex of awake primates.
As a MD/PhD student, I attempt to maintain a potential to adapt the end-products of each development for clinical applicability.
The story presented here is rather unusual in that success precedes failure.
The volume of tangible presentable results is greatest toward the beginning stages of the work described here.
This unusual inversion is what make this story worth hearing, however.
Thank you for taking the time to read this.
I hope that at least the technical information provided herein, if not the procedural insight, is valuable in your current or future endeavors.

\cleardoublepage

% The acknowledgment page should go here. Use something like
% \newpage\section*{Acknowledgments} followed by your text.
\newpage
\section*{\centerline{Acknowledgments}}
The support and patience I have received from my committee has gone far beyond what should be expected of anyone.
I can't thank you enough.
\begin{itemize}
	\item Xue Han, Ph.
	      D.
	\item David Boas, Ph.
          D.
    \item Kamal Sen Ph.D.
	\item Jerome Mertz, Ph.
	      D.
	\item Ian Davis, Ph.
	      D.
	\item Vickery Trinkaus-Randall, Ph.
	      D.
	\item Steven Borkan, M.
	      D.
\end{itemize}

\vskip 1in

\cleardoublepage

% The abstractpage environment sets up everything on the page except
% the text itself.  The title and other header material are put at the
% top of the page, and the supervisors are listed at the bottom.  A
% new page is begun both before and after.  Of course, an abstract may
% be more than one page itself.  If you need more control over the
% format of the page, you can use the abstract environment, which puts
% the word "Abstract" at the beginning and single spaces its text.

\begin{abstractpage}
    % ABSTRACT

Imaging of multiple cells has rapidly multiplied the rate of data acquisition as well as our knowledge of the complex dynamics within the mammalian brain.
The process of data acquisition has been dramatically enhanced with highly affordable, sensitive image sensors enable high-throughput detection of neural activity in intact animals.
Genetically encoded calcium sensors deliver a substantial boost in signal strength and in combination with equally critical advances in the size, speed, and sensitivity of image sensors available in scientific cameras enables high-throughput detection of neural activity in behaving animals using traditional wide-field fluorescence microscopy.
However, the tremendous increase in data flow presents challenges to processing, analysis, and storage of captured video, and prompts a reexamination of traditional routines used to process data in neuroscience and now demand improvements in both our hardware and software applications for processing, analyzing, and storing captured video.
This project demonstrates the ease with which a dependable and affordable wide-field fluorescence imaging system can be assembled and integrated with behavior control and monitoring system such as found in a typical neuroscience laboratory.

An Open-source MATLAB toolbox is employed to efficiently analyze and visualize large imaging data sets in a manner that is both interactive and fully automated.
This software package provides a library of image pre-processing routines optimized for batch-processing of continuous functional fluorescence video, and additionally automates a fast unsupervised ROI detection and signal extraction routine.
Further, an extension of this toolbox that uses GPU programming to process streaming video, enabling the identification, segmentation and extraction of neural activity signals on-line is described in which specific algorithms improve signal specificity and image quality at the singe cell level in a behaving animal.
This  project describes the strategic ingredients for transforming a large bulk flow of raw continuous video into proportionally informative images and knowledge.


%

\end{abstractpage}
\cleardoublepage

% Now you can include a preface. Again, use something like
% \newpage\section*{Preface} followed by your text

% Table of contents comes after preface
\tableofcontents
\cleardoublepage

% If you do not have tables, comment out the following lines
\newpage
\listoftables
\cleardoublepage

% If you have figures, uncomment the following line
\newpage
\listoffigures
\cleardoublepage

% List of Abbrevs is NOT optional (Martha Wellman likes all abbrevs listed)
\chapter*{List of Abbreviations}
\begin{center}
    \begin{tabular}{lll}
        \hspace*{2em}    & \hspace*{1in} & \hspace*{4.5in}                        \\
        CAD              & \dotfill      & Computer-Aided Design                  \\
        DOG              & \dotfill      & Difference Of Gaussian (distributions) \\
        FWHM             & \dotfill      & Full-Width at Half Maximum             \\
        LGN              & \dotfill      & Lateral Geniculate Nucleus             \\
        PDF              & \dotfill      & Probability Distribution Function      \\
        $\mathbb{R}^{2}$ & \dotfill      & the Real plane                         \\
    \end{tabular}
\end{center}
\cleardoublepage

% END OF THE PRELIMINARY PAGES

\newpage
\endofprelim
