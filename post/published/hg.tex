\section*{Unique contributions of parvalbumin and cholinergic interneurons in organizing striatal networks during movement}
\label{sec:hg}

Striatal parvalbumin (PV) and cholinergic interneurons (CHIs) are poised to play major roles in behavior by coordinating the networks of medium spiny cells that relay motor output.
However, the small numbers and scattered distribution of these cells have hindered direct assessment of their contribution to activity in networks of medium spiny neurons (MSNs) during behavior.
Here, we build on recent improvements in single-cell calcium imaging combined with optogenetics to test the capacity of PVs and CHIs to affect MSN activity and behavior in mice engaged in voluntary locomotion.
We find that PVs and CHIs have unique effects on MSN activity and dissociable roles in supporting movement.
PV cells facilitate movement by refining the activation of MSN networks responsible for movement execution.
CHIs, in contrast, synchronize activity within MSN networks to signal the end of a movement bout.
These results provide new insights into the striatal network activity that supports movement.

\noindent
Published 2019 \cite{Gritton_2019}
\clearpage

