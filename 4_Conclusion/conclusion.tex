\chapter{Discussion}
\label{chapter:discussion-the-last-mile-in-computing-for-clinicians-engineers-and-research-scientists}
\thispagestyle{myheadings}


\textbf{the last mile in computing for clinicians,
engineers, and research
scientists...}

This dissertation presents straightforward and reproducible methods for assembling laboratory equipment that can capture the behavior and neural activity of laboratory animals as well as the procedures for managing and analyzing the collected data. In some ways, the recommended procedures deviate from standard practice or the most obvious approaches. In this section, the newer approaches are compared and contrasted with current or traditional ones.

\section{State of current methods}\label{state-of-current-methods}

At the start of the work described here, we found ourselves with technology providing “neural signals” that vastly exceeded our expectations and the assumptions of the tools we applied to work with it. In the past, fluctations in optical imaging data were dominated by “noise.” The form of noise depended on the process; all types of imaging, intrinsic signal, fluorescent dye, etc., had relatively small fluctuations resulting from neural activity. With new engineered molecules, like GCaMP6, and new images sensors, like those dubbed scientific CMOS, these sources of noise were comparatively small. This improved signal-to-noise ratio opens the door for new opportunities and facilitates change to traditional analytic routines. The abundance of signals available from our research animals not only makes old routines inefficient, but paradoxically, also insufficient. Such an abundance of data factors at our finger tips requires a level of discipline in study design to make the scientific method work that was previously unnecessary as the difficulty in finding signals was inherently “self regulating” and inherently limiting.

\subsection{Signal and Noise in Neural Imaging
Data}\label{signal-and-noise-in-neural-imaging-data}

Traditional noise in neural signals can be roughly categorized as having origin in physiology or technology. The physiological noise sources include “artifacts” caused by an animal’s breathing, heart beat, or other physical movements in response to the experimentally controlled world around them. Technological noise is usually broken down into sensor noise sources:  read noise and thermal noise, and noise relating to digitization. A third type of “noise” could arguably be categorized as either, as it lies at the interface of technology and biology. For example, the complex interactions of exogenous calcium-binding proteins like GCaMP with the endogenous calcium handling proteins of a neuron potentially creates noise at the technology-biology interface. By strict definition, however, only the sensor noise should termed noise, as other sources are mostly predictable and unpredictable and can be systematically neutralized or accommodated prior to data analysis.
The noise level in the signals gathered by a combination of GCaMP6 and a sCMOS camera is miniscule relative to the signals indicating fluctuation in calcium concentration. The problem of visualization of these signals persists however, as the dynamic range of signal varies tremendously over space and time, and requires some treatment prior to being displayed on our currently limited computer monitors. Previously common methods, particularly intrinsic-signal imaging, provided very small signals that required “averaging over time” before any specific or reproducible response could be ascertained.


\subsection{Correlation, Confounding Signals, and Non-linear
dyanamics}\label{correlation-confounding-signals-and-non-linear-dyanamics}

\subsection{Motion Artifact}\label{motion-artifact}

\section{Exponential Expansion in Data
Volume}\label{exponential-expansion-in-data-volume}

The quality of cheaply available image sensors has risen drastically and are readily available. A workable interface can be readily established and the stream of information they provide once switched on is virtually unlimited. In the stark contrast however, storage for this never-ending data stream is both finite in its capacity, and cumulative in its consumption of available storage devices.

\subsection{Fields sharing these challenges}\label{fields-sharing-this-challenges}

Scientists often view themselves as working inside laboratory full of sensors, being “data-rich” but “space-poor”. For better or worse, scientists are not alone in dealing with this inherent technologic problem. Massive investment has been poured into managing this issue for commercial purposes, and – perhaps unsettlingly – for governmental surveillance purposes. The volume of recordable traffic bouncing through choke points of the internet exceeds the capacity of any government to store for more than about 24 hours. Likewise, the massive volume of video data acquired by video surveillance systems in China requires a similar solution to one desired by scientists and physicians to resolve our data acquisition challenges.

\subsection{Technological Opportunities to
Expect}\label{technological-opportunities-to-expect}

Current solutions proposed by commercial and governmental giants are not radical. They include calls for standardization in data format that could enable solutions for efficient transmission and storage to be shared by improving common tools. Common streaming formats allow compression and storage to be abstracted from each application. Databases are being developed to take advantage of heterogeneous computational architectures and distributed storage spaces. Traditional document-based or relational databases are outperformed by graph-based “triple-store” databases, timeseries databases, and by databases programmed for specific architecture, including GPU-databases. These technologic developments are targeted at the bottlenecks currently restricting access to data. Early results with these approaches suggest an orders of magnitude improvement in throughput. These tools are being developed both with and without the contribution of physicians and scientists. It would be prudent however, to take advantage of new developments by orienting these tools to the specific needs of scientists and clinicians.

\section{Clinical translation
potential}\label{clinical-translation-potential}

Devices that rely on optogenetics to deliver stimulation to neurons inherently share the same hurdles to clinical translation. These hurdles include the requirement for gene-therapy and its associated risks. Several early trials of viral transfection of cells had adverse effects including a greatly increased risk of carcinoma. In these early studies, the DNA insertion location was uncontrolled leaving important regions of DNA tumor suppressor genes exposed to damage. New methods that improve the safety of gene therapy have been developed. Several of the more recent methods utilize adeno-associated virus (AAV) with greater control regarding the site of DNA insertion and also cause less DNA damage. These more recent methods suggest the possibility that with continuing research, methods may be developed without the inherent potential to stimulate malignancy.
Working on a project that requires a technology that does not as of yet exist represents one of the greatest educational challenges and benefits of this project. That leap of faith into a future that also does not exist requires us to depend on each other as a team of collaborators in a mutually interdependent manner. In order to succeed, we must do so together. Without each other, our therapeutics would never reach their ultimate “target audience”, the patient. In this scenario, we share both successes and setbacks in the same meaningful way whether such events occur within our own labs or others located elsewhere.


\section{Cranial Window}\label{cranial-window}

The two-stage cranial implant device described here was developed to enable reliable, long-term optical access and intermittent physical access to mouse neocortex. Our particular application required bilateral cortical windows compatible with wide-field imaging through a fluorescence microscope, and physical access to the underlying tissue for virus-mediated gene delivery and injection of exogenous labeled cells. Optical access is required as soon as possible post-installation and ideally, is sustainable for several months thereafter. My current designs are focused on addressing the issue common to other window designs meant for rodents, that is, progressive degradation of the optical light-path at the brain-to-window interface caused by highly scattering tissue growth. The elastomer insert is molded to fit the chamber and craniotomy site, blocking tissue growth in way that provides a reliable optical interface lasting up to one year. Additionally, the core design can be rapidly adapted to improve its performance or interface with diverse applications.

\subsection{“Biomimicry” in visual processing}
This section describes how computer image and video processing relate to visual processing in the mammalian brain. The overall goal is to emphasize the advantage and importance of biomimetic development. Neuromorphic computing with “on chip” image processing repsents an improvement over “edge computing”.  Event-based image sensors such as the “artificial retina”or tittto attempt to replicate physiologic environments  wherein event streams are demanded that are asynchronous and threshold-based. Convolutional neural-nets and deep learning for specific tasks have secveal but also have substantial technlgoic similarities differences. Genetic programming approaches to procedure optimization will hopefully minimize latency whle maximizing sensitivity and accuracy at minimal  computational cost, energy expenditure (i.e., with high metabolic efficiency) that facilitates visual stream processing amenable to feature extraction with motion estimation and compensation. The current asymmetry of learning/training time negatively impacts on the  desired inference computation time. Common standards applied across projects with common themes can be facilitatied by employing rigorous adherence to non-proprietary open source conventions that includes (bt is not limited to) optical parts (lens threads), file formats, widely available software libraries in standard programming languages, and ease of file transmission that are web-based. We are well advised to borrow from related sectors with better developed solutions such as surveillance, media streaming for web/entertainment, sports, astronomy/telescopes, medical imaging and even automotive applications.




\subsection{Critical Elements}\label{critical-elements}

In assessing the design presented here, we highlight a few critical elements that facilitate the maintenance of the long-term optical quality. The Methods section describes the specifics of surgical procedures for headplate installation and insert attachment. These procedures were established after testing the variable formulations in protocol.
First, the design of the silicone insert must incorporate a mechanical barrier that fits along the edges of the craniotomy. To be effective, the barrier must be continuous along the circumference, and extend as far as the inside surface of the skull.
Achieving this tight fit without agressively impinging on the brain requires some sort of fine height adjustment capability. The silicone insert must be attached at the correct height during the installation procedure, or shortly thereafter. The insert must be slightly depressed until full contact is made across the entire window. However, pressing beyond this distance quickly exerts an untoward increase in intracranial pressure that promotes both inflammation and adverse outcomes. A mechanism for fine adjustment can be designed into the system and is in fact incorporated into the installation procedure as is done in the first design and demonstrated in the second design presented here.
Of particular note, we found that administration of antibiotic and anti-inflammatory drugs in the days surrounding any major surgical procedure had a substantial impact on the viability of the optical interface. We used both corticosteroid and non-steroidal anti-inflammatory drugs. Attempts to exclude either drug caused poor outcomes for study animals.
Lastly, sealing the chamber is absolutely critical for achieving viability of the optical interface as well as the animal’s overall well being. Equally critical to the long-term health of the imaging chamber is the requirement to establish and maintain an air-tight seal between the chamber and the outside world. This includes a permanent seal between the chamber and skull and a reversible seal between the chamber rim and the optical insert. is Although specific to the system design, a permanent seal is absolutely essential to ensure long-term functionality.
In addition to establishing and maintaining an air-tight seal, it is necessary to eliminate all air pockets within the chamber. Residual air pockets will be susceptible to bacteria growth and may disrupt normal intracranial and intermembrane pressures after installation. We employed sterile agarose fill to displace all air within the chamber prior to sealing. Dead space surrounding the silicone insert, including that temporarily filled with agarose, will fill with fluid and eventually be overtaken by granulation tissue. This preventive process is helpful to the maintenance of a sterile chamber environment and therefore, care should be taken not to disrupt it. However, an excess of dead space will delay this process and thus should also be minimized when adapting the design.
Several attempts to test variations from the described procedures indicated that all elements mentioned above are equally critical to achieving a reliable imaging window with sustained optical quality. Implementing the procedures as described or an effective alternative solution should mitigate the primary obstacle to long-term imaging in mice and other rodents by reducing the need to pre-terminate imaging experiments due to light-path disruption by tissue ingrowth. This capability will drastically reduce wasted time and resources for experiments of any duration and facilitate previously infeasible research that require longer-term observations including aging or progressive chronic neurological disorders.


\subsection{Staging Implant Installation \& Tissue
Access}\label{staging-implant-installation-tissue-access}

Configuring the implant as described to enable a staged installation of multiple parts enables surgical procedures to be easily spread across multiple days. This capability offers a number of advantages including saving time and resources, particularly during the prototype stages by allowing time to ensure each implanted animal fully recovers from the initial procedure and anesthesia. Additionally, the delay between surgeries allows the initial inflammation and immune system responses triggered by craniotomy to resolve before attempting a second intervention in tissue that is sensitive to these manipulations (e.g., viral or cell injections). Importantly, this system affords the capability to image the first tissue intervention from day 0. 
Similarly, designing a system installed in multiple stages enables trivial and repeatable tissue access at later time points by simply reversing the insert attachment procedure. The process may be comparable to a previously reported method of removing the entire cranial glass window to access the tissue.  With this newer system however, the methods used to detach and reattach the cranial window are relatively faster, simpler and carry less risk of tissue damage. Additionallly, the described methodd provide full cranial access without compromising the image field, an advantage not provided by a fixed access port. 


\subsection{Design Adaptation}\label{design-adaptation}

The specific designs described in this report work well and have much to offer.  The potential for fast and unrestricted adaptation is the greatest asset of the underlying system. Most users will find greater utility in adopting components of the design and fabrication process that can be readily customized to fit their exact needs. The design can also be rapidly transformed to accommodate various applications or to modify its performance in response to new technologies and demands. This rapid adaptability was a primary goal of this projec, and informed our design and engineering decisions throughout developement. Anyone with access to common laboratory equipment and moderate engineering and fabrication skills can produce a system to fit their particular needs. As an inherent aspect of any design process, the adaptation of the original design evolved over the course of prototyping and testing. In presenting two designs in this report, our intention was to demonstrate the technical feasibility of continuous development of a “future-proof” system. The original system was adapted to accomodate the continuous evolution of image sensor technology, particularly the growth in size and resolution, expanding the field of view and allowing simultaneous access to cellular interactions across multiple brain regions using wide-field imaging.
We found that subtle dimensional changes, and the addition of miniscule features exert a large impact on the success of any design. We also found that adjusting features to address one aspect of functionality may have unintended effects. For example, the inclusion of a thin skirt extending below the optical insert that was incorporated to protect against tissue growth within the image field also promotes physical conformity of the brain to the optical window interface over time. This conformity results in a flat imaging plane that is optimal for wide-field imaging and was previously unachievable. 

\subsection{Rapid fabrication}\label{rapid-fabrication}
The rapid iterative process used here was made possible by using a combination of widely available rapid prototyping procedures, 3D-printing and laser-cutting.  Major progress of manufacturing and its increased versatility, providing better quality, customization, lower cost and shorter production time. In an effort to compare various manufacturing technologies, we explored a number of companies and advanced with 3D metal printing. The final products provided by at i.materialise achieved our experimental goals in product design. We also developed parts in collaboration with other rapid prototyping companies including Shapeways and Sculpteo. In addition:
\begin{itemize}
	\item	Various features and functions of the silicone insert were transformed and extended to conform to new design requirements, some requiring distinctively different design approaches
	\item	Versatility of silicone elastomer to cover a spectrum of design strategies to optimize its configuration might be beneficial 
	\item	The design principles that evolved from the initial development are robust and can be applied to new developments or refinements while preserving the successful qualities of the original implant
	\item	CAD designs of these reported systems are open source accessible and can be modified and extended by evolving demands and technologies
We, the authors, also call for replication, adaptation, and evaluation (i.e., continued open/shared development).
\end{itemize}
\subsection{Future improvements}
The current project primarily explores the ability to mold precise and complex features using silicone elastomer to discover configurations to improve image performance using encapsulated electrodes and optical guides. These approaches replace combination optical + integrated electrode window and do not require optogenetics stimulation. More significantly, the encapsulation of carbon, metal colloidal particles or quantum dots into polymer hydrogel networks imparts exclusive thermal, sonic, optical, electrical or magnetic properties. Specifically, the polymer interface may provides a means for directly penetrating neurons to gain electrophysiological recording or facilitate drug infusions, allowing recording and/or manipulation during imaging session. In the near future, improvments in window thickness and chromatic aberration wil enhance both wider-field and 2-photon imaging, aprocess that will be enhanced by improved and lenses and embeded, integrated electronic components, such as LEDs for illumination or stimulation, or sensors.  These mbedded devices will facilitate positioning, especially in combination with kinematic headplates that alow for repeatable head positioning and newer fabrication materials.



\clearpage


% set this to the location of the figures for this chapter. it may
% also want to be ../Figures/2_Body/ or something. make sure that
% it has a trailing directory separator (i.e., '/')!
\graphicspath{{3_Conclusion/Figures/}}

\section{Summary of the thesis}

Time to get philosophical and wordy.

IMPORTANT: In the references at the end of thesis, all journal names must be
spelled out in full, except for standard abbreviations like IEEE, ACM, SPIE,
INFOCOM, ...